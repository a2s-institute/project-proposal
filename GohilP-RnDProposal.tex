\documentclass[rnd]{mas_proposal}
% \documentclass[thesis]{mas_proposal}

\usepackage[utf8]{inputenc}
\usepackage{amsmath}
\usepackage{amsfonts}
\usepackage{amssymb}
\usepackage{graphicx}

\title{Multimodal Machine Learning for Robotics}
\author{Priteshkumar Gohil}
\supervisors{Santosh Thoduka\\Prof. Dr. Paul G. Plöger}
\date{April 2019}

% \thirdpartylogo{path/to/your/image}

\begin{document}

\maketitle

\pagestyle{plain}

\chapter{Introduction}
\begin{itemize}
%	Introduction to the general topic i'm covering
    \item The environment that we experience is multimodal because we use visual, audio, touch, taste and smell sensory information. A very good example could be a person talking to someone will combine audio (voice) with the visual (lips movement) information to understand them better
    \item The robot is equipped with multiple sensors to measure an aspect of its environment and perform various task. 
    \item The measurement of environmental aspects from the sensors are correlated by some factors. For example, pick and place task can be performed better if we combine information from video and other manipulator sensors.
    \item According to Merriam Webster, vision is the prime part of sensation in modality.
    \item We focus to combine, process and relate multiple sensory modalities using machine learning techniques to solve a certain task in the field of robotics.
    \item Developing such a model can help to correlate visual information with multiple other modalities from robot sensors to understand the robot's environment better and increase the performance.
    
%    relevance
    \item The approach can help robot to understand it's environment better or accurately given a particular task.
	\item It can be applied to following task performed by the robot,
	\begin{itemize}
		\item Pick and place task
		\item Fault detection and diagnosis
		\item Audio-Visual Speech recognition
		\item Event recognition
	\end{itemize}
	\item Other than Robotics, multimodal machine learning can also be used for,
	\begin{itemize}
		\item In the medical field such as neuroimaging to fuse data from EEG, MEG, MRI and increase the performance of diagnosis \cite{Daehne2015}.
		\item Media captioning
	\end{itemize}
\end{itemize}

\section{Problem Statement}
\begin{itemize}
    \item What are you going to solve?
    \item How are you evaluating?
\end{itemize}


\chapter{Related Work}
\begin{itemize}
    \item What have other people done?
    \item Why is it not sufficient?
\end{itemize}

\section{Section 1}
\section{Section 2}



\chapter{Project Plan}

\section{Work Packages}
The bare minimum will include the following packages:
\begin{enumerate}
    \item[WP1] Literature Search
    \item[WP2] Experiments
    \item[WP3] Project Report
\end{enumerate}
Keep in mind that depending on your project, you will probably need to add work packages that are more suited to your projects.

\section{Milestones}
\begin{enumerate}
    \item[M1] Literature search
    \item[M2] Experimental setup
    \item[M3] Experimental Analysis
    \item[M4] Report submission
\end{enumerate}

\section{Project Schedule}
Include a gantt chart here. It doesn't have to be detailed, but it should include the milestones you mentioned above.
Make sure to include the writing of your report throughout the whole project, not just at the end.

\begin{figure}[h!]
    \includegraphics[width=\textwidth]{rnd_deliverable_timeline}
    \caption{}
    \label{}
\end{figure}

\section{Deliverables}
\subsection{Minimum Viable}

\begin{itemize}
    \item Survey
    \item Analysis of state of the art
    \item Simple simulated use case
    \item Demo on youBot or Jenny
\end{itemize}

\subsection{Expected}
\begin{itemize}
    \item Comparation of approaches in the robot
\end{itemize}

\subsection{Desired}
\begin{itemize}
    \item Integration to scenario
\end{itemize}


\nocite{*}

\bibliographystyle{plainnat} % Use the plainnat bibliography style
\bibliography{bibliography.bib} % Use the bibliography.bib file as the source of references




\end{document}
